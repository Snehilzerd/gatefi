\let\negmedspace\undefined
\let\negthickspace\undefined
\documentclass[journal,12pt,onecolumn]{IEEEtran}
\usepackage{setspace}
\singlespacing
\usepackage[cmex10]{amsmath}
\usepackage{amsthm}
\usepackage{mathrsfs}
\usepackage{txfonts}
\usepackage{stfloats}
\usepackage{bm}
\usepackage{cite}
\usepackage{cases}
\usepackage{subfig}
\usepackage{longtable}
\usepackage{multirow}
\usepackage{enumitem}
\usepackage{mathtools}
\usepackage{tikz}
\usepackage{circuitikz}
\usepackage{verbatim}
\usepackage[breaklinks=true]{hyperref}
\usepackage{tkz-euclide} % loads  TikZ and tkz-base
\usepackage{listings}
\usepackage{color}    
\usepackage{array}    
\usepackage{longtable}
\usepackage{calc}     
\usepackage{multirow} 
\usepackage{hhline}   
\usepackage{ifthen}   
\usepackage{lscape}     
\usepackage{chngcntr}
\DeclareMathOperator*{\Res}{Res}
\renewcommand\thesection{\arabic{section}}
\renewcommand\thesubsection{\thesection.\arabic{subsection}}
\renewcommand\thesubsubsection{\thesubsection.\arabic{subsubsection}}
\renewcommand\thesectiondis{\arabic{section}}
\renewcommand\thesubsectiondis{\thesectiondis.\arabic{subsection}}
\renewcommand\thesubsubsectiondis{\thesubsectiondis.\arabic{subsubsection}}
\renewcommand{\thefigure}{\theenumi}
\renewcommand{\thetable}{\theenumi}
\providecommand{\gauss}[2]{\mathcal{N}\ensuremath{\left(#1,#2\right)}}
% correct bad hyphenation here
\hyphenation{op-tical net-works semi-conduc-tor}
\def\inputGnumericTable{}                                 %%

\lstset{
%language=C,
frame=single, 
breaklines=true,
columns=fullflexible
}
%\lstset{
%language=tex,
%frame=single, 
%breaklines=true
%}
\begin{document}
\newtheorem{theorem}{Theorem}[section]
\newtheorem{problem}{Problem}
\newtheorem{proposition}{Proposition}[section]
\newtheorem{lemma}{Lemma}[section]
\newtheorem{corollary}[theorem]{Corollary}
\newtheorem{example}{Example}[section]
\newtheorem{definition}[problem]{Definition}
\newcommand{\BEQA}{\begin{eqnarray}}
\newcommand{\EEQA}{\end{eqnarray}}
\newcommand{\define}{\stackrel{\triangle}{=}}
\bibliographystyle{IEEEtran}
\providecommand{\mbf}{\mathbf}
\providecommand{\pr}[1]{\ensuremath{\Pr\left(#1\right)}}
\providecommand{\qfunc}[1]{\ensuremath{Q\left(#1\right)}}
\providecommand{\sbrak}[1]{\ensuremath{{}\left[#1\right]}}
\providecommand{\lsbrak}[1]{\ensuremath{{}\left[#1\right.}}
\providecommand{\rsbrak}[1]{\ensuremath{{}\left.#1\right]}}
\providecommand{\brak}[1]{\ensuremath{\left(#1\right)}}
\providecommand{\lbrak}[1]{\ensuremath{\left(#1\right.}}
\providecommand{\rbrak}[1]{\ensuremath{\left.#1\right)}}
\providecommand{\cbrak}[1]{\ensuremath{\left\{#1\right\}}}
\providecommand{\lcbrak}[1]{\ensuremath{\left\{#1\right.}}
\providecommand{\rcbrak}[1]{\ensuremath{\left.#1\right\}}}
\theoremstyle{remark}
\newtheorem{rem}{Remark}
\newcommand{\sgn}{\mathop{\mathrm{sgn}}}
\providecommand{\abs}[1]{\left\vert#1\right\vert}
\providecommand{\res}[1]{\Res\displaylimits_{#1}} 
\providecommand{\norm}[1]{\left\lVert#1\right\rVert}
\providecommand{\mtx}[1]{\mathbf{#1}}
\providecommand{\mean}[1]{E\left[ #1 \right]}
\providecommand{\fourier}{\overset{\mathcal{F}}{ \rightleftharpoons}}
\providecommand{\system}[1]{\overset{\mathcal{#1}}{ \longleftrightarrow}}
\newcommand{\solution}{\noindent \textbf{Solution: }}
\newcommand{\cosec}{\,\text{cosec}\,}
\providecommand{\dec}[2]{\ensuremath{\overset{#1}{\underset{#2}{\gtrless}}}}
\newcommand{\myvec}[1]{\ensuremath{\begin{pmatrix}#1\end{pmatrix}}}
\newcommand{\mydet}[1]{\ensuremath{\begin{vmatrix}#1\end{vmatrix}}}
\let\vec\mathbf
\def\putbox#1#2#3{\makebox[0in][l]{\makebox[#1][l]{}\raisebox{\baselineskip}[0in][0in]{\raisebox{#2}[0in][0in]{#3}}}}
     \def\rightbox#1{\makebox[0in][r]{#1}}
     \def\centbox#1{\makebox[0in]{#1}}
     \def\topbox#1{\raisebox{-\baselineskip}[0in][0in]{#1}}
     \def\midbox#1{\raisebox{-0.5\baselineskip}[0in][0in]{#1}}
\setlength{\parindent}{0pt}
\bibliographystyle{IEEEtran}
\newenvironment{amatrix}[1]{%
  \left(\begin{array}{@{}*{#1}{c}|c@{}}
}{%
  \end{array}\right)
}
\title{
%	\logo{
GATE BM 2023 Quetion 9
%	}
}
\author{ Snehil Singh EE22BTECH11050$^{*}$% <-this % stops a space
	%
}
	
	
%\title{
%	\logo{Matrix Analysis through Octave}{\begin{center}\includegraphics[scale=.24]{tlc}\end{center}}{}{HAMDSP}
%}


% paper title
% can use linebreaks \\ within to get better formatting as desired
%\title{Matrix Analysis through Octave}
%
%
% author names and IEEE memberships
% note positions of commas and nonbreaking spaces ( ~ ) LaTeX will not break
% a structure at a ~ so this keeps an author's name from being broken across
% two lines.
% use \thanks{} to gain access to the first footnote area
% a separate \thanks must be used for each paragraph as LaTeX2e's \thanks
% was not built to handle multiple paragraphs
%
%\author{<-this % stops a space
%\thanks{}}
%}
% note the % following the last \IEEEmembership and also \thanks - 
% these prevent an unwanted space from occurring between the last author name
% and the end of the author line. i.e., if you had this:
% 
% \author{....lastname \thanks{...} \thanks{...} }
%                     ^------------^------------^----Do not want these spaces!
%
% a space would be appended to the last name and could cause every name on that
% line to be shifted left slightly. This is one of those "LaTeX things". For
% instance, "\textbf{A} \textbf{B}" will typeset as "A B" not "AB". To get
% "AB" then you have to do: "\textbf{A}\textbf{B}"
% \thanks is no different in this regard, so shield the last } of each \thanks
% that ends a line with a % and do not let a space in before the next \thanks.
% Spaces after \IEEEmembership other than the last one are OK (and needed) as
% you are supposed to have spaces between the names. For what it is worth,
% this is a minor point as most people would not even notice if the said evil
% space somehow managed to creep in.



% The paper headers
%\markboth{Journal of \LaTeX\ Class Files,~Vol.~6, No.~1, January~2007}%
%{Shell \MakeLowercase{\textit{et al.}}: Bare Demo of IEEEtran.cls for Journals}
% The only time the second header will appear is for the odd numbered pages
% after the title page when using the twoside option.
% 
% *** Note that you probably will NOT want to include the author's ***
% *** name in the headers of peer review papers.                   ***
% You can use \ifCLASSOPTIONpeerreview for conditional compilation here if
% you desire.




% If you want to put a publisher's ID mark on the page you can do it like
% this:
%\IEEEpubid{0000--0000/00\$00.00~\copyright~2007 IEEE}
% Remember, if you use this you must call \IEEEpubidadjcol in the second
% column for its text to clear the IEEEpubid mark.
% make the title area 
\maketitle


%\tableofcontents

\bigskip

\renewcommand{\thefigure}{\arabic{figure}}
\renewcommand{\thetable}{\theenumi}
%\renewcommand{\theequation}{\theenumi}

%\begin{abstract}
%%\boldmath
%In this letter, an algorithm for evaluating the exact analytical bit error rate  (BER)  for the piecewise linear (PL) combiner for  multiple relays is presented. Previous results were available only for upto three relays. The algorithm is unique in the sense that  the actual mathematical expressions, that are prohibitively large, need not be explicitly obtained. The diversity gain due to multiple relays is shown through plots of the analytical BER, well supported by simulations. 
%
%\end{abstract}
% IEEEtran.cls defaults to using nonbold math in the Abstract.
% This preserves the distinction between vectors and scalars. However,
% if the journal you are submitting to favors bold math in the abstract,
% then you can use LaTeX's standard command \boldmath at the very start
% of the abstract to achieve this. Many IEEE journals frown on math
% in the abstract anyway.

% Note that keywords are not normally used for peerreview papers.
%\begin{IEEEkeywords}
%Cooperative diversity, decode and forward, piecewise linear
%\end{IEEEkeywords} 
\textbf{Question :}\\
Out of 1000 individuals in a town, 100 unidentified individuals are covid positive.
Due to lack of adequate covid-testing kits, the health authorities of the town devised
a strategy to identify these covid-positive individuals. The strategy is to:
\begin{enumerate}
\item Collect saliva samples from all 1000 individuals and randomly group
them into sets of 5.
\item Mix the samples within each set and test the mixed sample for covid.
\item If the test done in (ii) gives a negative result, then declare all the 5
individuals to be covid negative.
\item If the test done in (ii) gives a positive result, then all the 5 individuals
are separately tested for covid.
\end{enumerate}
Given this strategy, no more than testing kits will be required to identify
all the 100 covid positive individuals irrespective of how they are grouped \\ 
\begin{enumerate}
\item 700
\item 600
\item 800
\item 1000
\end{enumerate}

\textbf{Solution :}


Given : 
\begin{table}[h]
\def\arraystretch{1.2}
\begin{tabular}{|c|c|}
\hline
	\textbf{Parameter} &\textbf{Value} \\ \hline
	Number of individuals &1000 \\ \hline
	Strenght of each group &5 \\ 
	\hline
	Number of groups &200 \\ 
	\hline
	Number of Covid positive individuals &100 \\ \hline
	
	
\end{tabular}
\end{table}
\begin{align}
Number of tests initially to examine each group &=200
\end{align}
\begin{table}[h]
\def\arraystretch{1.2}
\begin{tabular}{|c|c|}
\hline
	\textbf{Probability} &\textbf{Value}  \\ \hline
	Probability of individual testing positive &0.1 \\ \hline
	Probability of individual testing negative  &0.9 \\ 
	\hline
	
	
\end{tabular}
\end{table}

\begin{table}[h]
\def\arraystretch{1.2}
\begin{tabular}{|c|c|c|}
\hline
	\textbf{Parameter} &\textbf{Value} &\textbf{Expression} \\ \hline
	n &5 &Number of individuals in group \\ \hline
	p &0.1 &Probability of testing positive \\ \hline
	q &0.9 &Probability of testing negative \\ \hline
	
	
\end{tabular}
\end{table}
Using binomial expansion to find the probability of a group testing positive
\begin{align}
Pr(X=i) &= ^{n}C_{i}(q)^{i}(p)^{n-i} \\
&= ^{5}C_{i}(0.9)^{i}(0.1)^{5-i}
\end{align}
And the CDF is
\begin{align}
F_{X}(k) &= Pr(X \le k) \\
&= \Sigma_{i=0}^k ^{n}C_{i}(q)^{i}(p)^{n-i} \\
&= \Sigma_{i=0}^k ^{5}C_{i}(0.9)^{i}(0.1)^{5-i} \\
\end{align}
Where i is number of individuals tested positive 
Evaluating probability of every case in a group
\begin{table}[h]
\def\arraystretch{1.2}
\begin{tabular}{|c|c|c|}
\hline
	\textbf{Parameter} &\textbf{Calculation} &\textbf{Probability} \\ \hline
	All positive &$^{5}C_{5}(0.1)^{5}$ &0.00001\\ \hline
	4 positive 1 negative &$^{5}C_{4}(0.1)^{4}(0.9)$ &0.00045  \\ 
	\hline
	3 positive 2 negative &$^{5}C_{3}(0.1)^{3}(0.9)^{2}$ &0.00810 \\ 
	\hline
	2 positive 3 negative &$^{5}C_{2}(0.1)^{2}(0.9)^{3}$ &0.07290\\ \hline
	1 positive 4 negative &$^{5}C_{1}(0.1)^{1}(0.9)^{4}$ &0.32805 \\ \hline
	All negaive &$^{5}C_{0}(0.9)^{5}$ &0.5904 \\ \hline
\end{tabular}
\end{table}
\begin{align}
P_{(of a group testing positive)} &= F_{X}(4) \\ 
&= \Sigma_{i=0}^4 ^{5}C_{i}(0.9)^{i}(0.1)^{5-i} \\
&= ^{5}C_{5}(0.1)^{5} + ^{5}C_{4}(0.1)^{4}(0.9) + ^{5}C_{3}(0.1)^{3}(0.9)^{2} + ^{5}C_{2}(0.1)^{2}(0.9)^{3} + ^{5}C_{1}(0.1)^{1}(0.9)^{4} \\
&= 0.00001 + 0.00045 + 0.0081 + 0.0729 + 0.32805 
\\ &= 0.4096
\end{align}

Now calculating the expected number of sets:
\begin{align}
Expected number of tests for one set &= 1 + 5*P_{(of a group testing positive)}  \\
&= 1 + 5*0.4096 \\
&= 3.048
\end{align}
\begin{align}
Total expected number of sets &= 200 * (Expected number of tests for one set) \\
&= 200 * 3.048 \\
&= 609.6
\end{align}

$\therefore$ So 609.6 is the number of tests required to detect all the covid positive individuals. \\
According to the options, no more than 700 tests will be required.

\textbf{So, the correct answer is option A 700}.


\end{document}
